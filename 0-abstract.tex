
% With the increasingly 

% Scientific computing 

% Scheduling 


% With an increasing interest in containerization technology, the .

% Scientific computing.
\documentclass[12pt]{article}
\begin{document}
% With a DAG-based scientific applications it has become quite a challenge to run them efficiently.

% An increasing number of modern scientific workflows are being run in containerized environments, such as Kubernetes.
% This is possible with workflow management systems controlling the deployment of specific tasks on the cluster.

% With the rising popularity of open-source platforms, such as Kubernets clusters, it becomes more and more ina target to more and more  transition to the common and supported ground is expected.

% While it enables the execution of DAG-based applications, it does not, however, answer nor focus in any way on the problem of efficiency.

Scheduling is an aspect of great concern for running scientific applications for they are often sizeable and computation heavy.
With an increasing number of modern scientific workflows being introduced and run in containerized environments, such as Kubernetes, it is important to consider the aspect of execution performance.
% In this thesis, the   is investigated and their performance is analyzed.
As most of the cluster schedulers, the Kubernetes one is only focused on allocating available resources for the requested workloads.
It is not aware of workflow task dependency, and its possibly locally optimal pod to resource assignment may lead to the long-term inefficient application execution.
To address this problem, the concept of workflow-aware workload scheduling for Kubernetes is proposed in this work.
% To address this problem, the topic of scheduling such applications in Kubernetes is investigated.
% As a result, the concept of workflow-aware workload scheduling for Kubernetes is proposed. % in this work
All discussed approaches are evaluated in the AWS cloud environment and their performance is compared across three different experimental scenarios, including task clustering.
Results show a decent improvement in makespan reduction with the newly introduced solution.
\end{document}