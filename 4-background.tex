
\section{Background} \label{section:background}

% WIP: W tym rozdziale opisane zostaną pojęcia ściśle powiązane z tematyką pracy. Zapoznanie się z nimi powinno ułatwić zrozumienie i analizę przygotowanego rozwiązania jak i eksperymentów. 

% While both represent a group of list-based scheduling algorithms

\subsection{Workflow}
\label{subsection:Workflow}

A workflow is a term used to describe a set of tasks, where some of them are dependent one on another and all need to be processed in order to produce final results.
Each job in such pipeline requires its own input data to begin its work and returns an output, when it ends successfully.
Furthermore, not all jobs do the same work -- they are often split by specific operations, which often allows for an easy and straightforward data parallelized job execution with ocassional synchronization \emph{(e.g. merge operation)}.    
As tasks may not start without their input, they might need to wait for an output of other tasks, which creates a dependency relationship.
Such pipeline can be represented in form of a \emph{Directed Acyclic Graph (DAG)}.
The following is a formal definition of workflow by XYZ:

% formal definition

An example scientific workflows are Montage \cite{Montage} and SoyKB \cite{SoyKB}.
Both of them were used in an experiment on scheduling further described in this work.


\subsubsection{Montage}


\subsubsection{SoyKB}


% Kubernetes
\subsection{Kubernetes}
\label{subsection:Kubernetes}

One of the most popular and advanced container management tools available to choose from is Kubernetes.
Being an open-source solution easily accessible for use in all kinds of projects -- scientific, commercial or private -- and allows for writing customs extensions to adjust its functionalities to a user needs.

\subsubsection{Architecture}


\subsubsection{Deployment}


\subsubsection{Pod}
asd asd asd


\subsubsection{Job}


\subsubsection{Kube-scheduler}


%WMS
\subsection{Workflow Management System}

asd asd asd

\subsection{Hyperflow}

% Workflow representation
% \begin{lstlisting}[language=json,firstnumber=1]
% {"menu": {
%   "id": "file",
%   "value": "File",
%   "popup": {
%     "menuitem": [
%       {"value": "New", "onclick": "CreateNewDoc()"},
%       {"value": "Open", "onclick": "OpenDoc()"},
%       {"value": "Close", "onclick": "CloseDoc()"}
%     ]
%   }
% }}
% 0123456789
% \end{lstlisting}

asd asd asd



% Scheduling
\subsection{Scheduling}
\label{section:Scheduling}

asd asd asd

\subsubsection{DAG Scheduling}

asd asd asd

\subsubsection{Cluster scheduling}

asd asd asd

\subsection{Cloud Computing}

asd asd asd



// \textbf{Do przeredagowania i uzupelnienia - struktura:}

\begin{itemize}
  \item Kubernetes (opis technologii i wyjasnienie pojec)
  \item Scheuduling (rodzaje, podejscia i algorytmy)
  \item Hyperflow (opis, architektura, definicja workflow)
\end{itemize}