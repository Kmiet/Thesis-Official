\subsection{Cloud computing}
\label{s:ProblemDomain:Cloud}

The term \emph{cloud computing} is used to describe a group of services with elastic pool of resources available to the end users \emph{on-demand} and settleable based on their usage quota \emph{(pay-as-you-go)} \cite{b:Cloud-Principles-Paradigms}.
To distinguish different cloud services, they are often classified by their abstraction level as specific service models with the three main being:

\begin{itemize}
  \item{
\emph{Software as a Service (SaaS)} -- the primary resources are the provider's applications that run and are managed transparently for the user
};
  \item{
\emph{Platform as a Service (PaaS)} -- enables the deployment of user applications on computing units which abstracts actual physical resources from the user, preventing any interaction with the underlying infrastructure
};
  \item{
\emph{Infrastructure as a Service (IaaS)} -- a model where users have to configure and manage the infrastructure by themselves, where the base resources are e.g. virtual machines, data storage and network components.
}
\end{itemize}
The abstraction level of a service is related to the number of intermediary layers between the final resource and the physical infrastructure.



\subsubsection{Container as a Service}
\label{s:ProblemDomain:CaaS}

Another model of cloud services available in today's clouds is \emph{Container as a Service (CaaS)}.
Relating it to other service models, it is classified as one in between of the IaaS and PaaS \cite{b:IBM-CaaS}.
With containers being treated as a primary resource, it enables interaction with the infrastructure at a higher abstraction level than in IaaS, nonetheless the responsibility for the management process still lies in hands of the end user.
This model offers a more scalable and portable solution to infrastructure management for containerized environments.



\subsubsection{Elastic Kubernetes Service}
\label{s:ProblemDomain:EKS}

An example of an CaaS service is \emph{Elastic Kubernetes Service (EKS)} \cite{b:IBM-CaaS} provided by \emph{Amazon Web Services (AWS)}.
It allows configuration of fully managed Kubernetes clusters \cite{b:AWS-EKS} in AWS cloud environment.
This means that EKS by itself spans a new cluster, covers the Kubernetes installation process, and is responsible for maintaining the Kubernetes control plane outside the cluster's nodes.
The end user is only responsible for initial cluster configuration and requesting workloads on the cluster with Kubernetes API.