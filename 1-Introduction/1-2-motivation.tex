\subsection{Motivation and problem statement}
\label{s:Introduction:Motivation}

% Wspominamy o zwiększonej liczbie badań o obliczeniach naukowych na Cloud i kontenerach
% -> odpowiednio Skyport, Magellan ->
% Przechodzimy na klastry kontenerowe w cloudzie
% -> Kubernetes ->
% zaczynamy o workflowach na Kubernetesie DAGi
% -> Hyperflow -> execution flow control system -> jak umozliwia wykonanie workflowów w K8s ->
% wspominamy, że występuje brak faktycznego optymalizacji / planowania DAG podczas schedulingu
% -> biorąc pod uwagę inne systemy WMS chemy zweryfikować przydatność takiego rozwiązania
% (info o korzystnym wplywie schedulingu na wykonanie obliczen DAG w WMS? -> Pegasus)
% i zweryfikować korzyści w nowym środowisku wykonawczym



As the interest in container computing increases continuously, increasingly more works about scientific computing begin to revolve around containerized execution environments \cite{b:Skyport, b:Hyperflow-K8s}. % \cite{b:Skyport}
The same phenomenon can be observed in the research area of cloud computing.
Due to the advantages of on-demand resource availability and scaling capabilities, studies are being made with a focus on e-science in the cloud environment context \cite{b:Magellan}. With this situation ongoing, Kubernetes has become one of the key platforms to run scientific applications for research purposes \cite{b:Enable-HPC-Cloud-K8s}.

An execution of a DAG-based scientific workflow in Kubernetes is not directly supported as Kubernetes provides only cluster management services.
The default cluster scheduler in Kubernetes considers a single workload to be a separate, containerized application and is only concerned with assigning resources for it to run on. %% do poprawy
Currently, to ensure a proper workflow parallelization, each task is delegated to a separate container and treated as an independent workload.
%
By leveraging the workflow management system engine, the execution flow of DAG-based applications is controlled \cite{b:Pegasus, b:Hyperflow}, ensuring the completion of preceding jobs before running a new one. 
% However it, still requires the scheduler to manage t
%
% Current solution for running such applications is to leverage a workflow management system engine to control the execution flow of the DAG-based application \cite{b:Pegasus, b:Hyperflow}.
%
% \cite{b:Hyperflow-k8s-deployment}
% by monitoring job statuses and allocating only the ready tasks.
However, in this approach it is not possible to leverage the known DAG structure of the application to plan the whole job execution before runtime in contrast to workflow scheduling solutions.
% However, this approach does not allow by itself to plan the whole job execution before runtime in contrast to workflow scheduling solutions.
As running such workloads efficiently is an important concern, for which workflow scheduling is considered as one of the key factors \cite{b:Grid-Workflow-Scheduling-Strategies}, the use of the current approach may now result in their suboptimal execution.
Introducing a new workflow-aware solution to scheduling DAG-based applications addresses this problem and may provide a possible makespan reduction and better load distribution across available resources.
% Not only a properly introduced solution ensures the proper workflow execution, taking into account its dependencies, it may also optimize the whole process for shortened makespan or better load distribution across available resources.

Nonetheless, the potential advantages of workflow-aware scheduling solutions for Kubernetes clusters are yet to be confirmed as, to our best knowledge, such scenarios have not been investigated so far. 

%% Cloud -> CaaS -> Kubernetes

