\subsection{Research methodology}\label{s:Introduction:Methodology}

% \todo{Describe methodology}

To assess the impact of scheduling dag-oriented workloads in Kubernetes clusters in terms of efficiency and performance
with analysis and comparison of available approaches,
a comparative experiment was conducted with a proposed \emph{proof-of-concept (PoC)} scheduler for Hyperflow engine.
For an evaluation of the introduced solution a few well-known and already established metric were selected, such as:

\begin{itemize}
  \item Makespan,
  \item Schedule Length Ratio,
  \item Job Slowdown.
\end{itemize}

The data gathered through experimental process came from execution of real-life scientific workflows,
namely Montage \cite{b:Montage} and SoyKB \cite{b:SoyKB-workflow-url}. % Change soyKB citation?
Comprising of logs and system metrics it included information about ..... 

and included timestamps for ....... .

Collected data was later processed and used for execution trace visualization and metric calculation.