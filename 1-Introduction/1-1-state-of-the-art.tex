% \subsection{State of the art}\label{s:Introduction:Art} % Containerization and e-Science

% \todo{Czy tytuł jest adekwatny? Użyć innego (Containerization and e-Science?) czy może bez podrozdziału?}

In recent times, scientific computing has become increasingly important.
The increase of worldwide gathered data allows to extract even more knowledge.
However, this often comes at the cost of a higher demand for computing power,
as the complexity of data processing steps grows.
To simplify the process, computations are often decomposed into workflows, specifying tasks that can be executed in parallel and dependencies between them.

% Managing the processes of job execution and exposing a common storage is a responsibility of workflow management systems.
% Each system usually provides a different a different 
% Pegasus 


% Workflows are pipelines comprised of interdependant tasks, required to be executed in a specified order to ensure successful job completion.
% Additionally, the tasks need to save their results in a common storage.
% These responsibilities are borne by workflow management systems.
% An example 
% Pegasus \cite{b:Pegasus}.
% Hyperflow is the other workflow management system 
% The Pegasus system ... cos o Pegasus i Hyperflow.

The workflow applications are often run in computing clusters.
Nowadays, after the paradigm shift towards virtualization, the primary cluster resources steadily change to virtual machines or containers.
To setup and monitor such infrastructures, cluster managers come to help, such as Borg developed by Google \cite{b:Borg}.
Another solution for cluster management is Kubernetes, which handles only containerized workloads \cite{b:Kubernetes-what-is}.
The suitability of containerized environments for scientific computing purposes has been studied in \cite{b:Enable-HPC-Cloud-K8s} by running HPC applications on various infrastructures, including a Kubernetes cluster spanned over cloud-provided virtual machines.
Modern scientific workflow management systems, such as Hyperflow, support such containerized infrastructures \cite{b:Hyperflow-K8s}.


Container solutions are tightly connected to the topic of cloud computing.
This comes to the point where a new, container-oriented model of cloud services -- CaaS \cite{b:IBM-CaaS} has been introduced.
Skyport \cite{b:Skyport} brings in a new approach to use containers for running scientific computations on resources from different providers, including computing clouds.
Considering both e-Science and cloud, it is worth mentioning that some of the workflow management systems support clouds as their execution environments
\cite{b:Pegasus}.
The applicability of the cloud as an execution environment for scientific workloads has been studied in in \cite{b:Magellan} and is mentioned to be dependable on application communication costs.


% 




%%%%%%%%% OSTATNIE
% Nauka -> workflowy -> WMS -> Pegasus -> Hflow -> gdzie odpalane? zazwyczaj klastry ->

% klastry -> wirtualizacja -> kontenery -> Borg -> Kubernetes ->

% -> K8s, kontenery i HPC, cloud -> Enabling K8s ->

% CaaS i Cloud -> Skyport -> nauka na cloudzie i kontenerach ->

% Magellan -> nauka w chmurze ważna sprawa
%%%%%%%%%


%%%%%%%%%
% Wirtualizacja i kontenery -> klastry -> Borg -> Kubernetes ->
% -> puszczanie HPC na K8s -> Enable-HPC-Cloud-K8s -> 
% -> ******
% -> naukowe rzeczy -> Skyport (Containers, Cloud, Scientific Workflows)
% -> naukowe rzeczy -> Magellan (Cloud, Science)
% -> scheduling różne podejścia, HPC -> Mesos?
%%%%%%%%%


%%%%%%%%%
% Zacznij od e-nauki -> workflowy -> Montage -> WMS -> Pegasus -> Hyperflow
% ->
% Wyjście do e-nauki + kontenery -> 
% nauka -> 
%%%%%%%%%


% \cite{b:Magellan} 22
% \cite{b:Borg} 26
% \cite{b:Borg-K8s-predecessor} 15
% \cite{b:Skyport} 9

% \cite{b:Grid-Workflow-Scheduling-Strategies} 27

% \cite{b:Enable-HPC-Cloud-K8s} 4

% \cite{b:Pegasus} 7 %% Task Clustering
% \cite{b:Hyperflow} 3

% \cite{b:Kubernetes-what-is} 17




% In the recent times a rise of an interest in a virtualization and containerization technologies had been seen.
% More and more focus has been put on an idea of software being run in an isolated and from the underlying physical environment.
% % T
% Many of that ideas had been also introduced in computing clouds to allow for better service separability and scalability.
% Additionally it allowed new services to be released, an example being the cloud functions used as the most cost and load adaptable form of computation.
% New services and systems which were being developed, mostly enterprise applications, started using the container-first approach and it has slowly became a standard for development an software design these days.
% % More and more people get itere

% The same trends apply to the field of scientific computing.
% Once strictly restricted to be ran on a massive clusters .
% Nowadays, more and more of scientific workloads are being slowly transited and adjusted for an execution in the highly virtualized environments, such as computing clouds, be it \emph{virtual machine (VM)} based clusters or ones managed by container orchestrator.
% With the recently gaining popularity Kubernetes \cite{Kubernetes} platform, a lot of approaches were made to adapt it for a possible scientific clusters management.
% One of a new workflow management systems that was designed with their engine being run in Kubernetes environment is Hyperflow \cite{HyperFlow}.


% %%%

% The problem is what you are going to investigate. The motivation is why. This usually narrows down to: (a) it would be useful or (b) it would fill in an area of knowledge.

% For example:

% Problem: I intend to measure the quantity of C produced when different amounts of B are added to A. This will be done in the presence of varying concentrations of D, E, and mixtures of them.

% Motivation: The production of C is of major importance in agriculture / neurochemistry / liquid crystal manufacture / the growth of slime on the bottoms of ships / teaching children / etc. It is generally understood to come from the action of B on A (key references). However, both D and E are sometimes present and may affect this process.(supporting references). As yet, there has been no systematic investigation of the possible interactions.

% Understanding this may lead to reduction in the use of pesticides / better control of neurological conditions / lower costs for liquid crystal production / a more consistent method of controlling drag on vessels / greater effectiveness in primary education for children who….

%%%




% The state of the art (sometimes cutting edge or leading edge) refers to the highest level of general
% development, as of a device, technique, or scientific field achieved at a particular time.

% https://www.researchgate.net/publication/332416844_Performance_Analysis_of_List_Scheduling_Algorithms_by_Random_Synthetic_DAGs
% ^ Do introduction przegladnij [1]