\subsection{Objectives}
\label{s:Introduction:Problem}

% \todo{will be?} WSZEDZIE CZAS PRZYSZŁY? do rozważenia

% \todo{Write a problem statement}

%% Investigate, analyze  wyznaczenie, wskazanie, wybór porównanie (choose, pick)

The problem of investigating the topic of scheduling scientific workflows in Kubernetes cluster revolves around the lack of already existing workflow-aware solutions.
Inspired by the idea of two-step scheduling process from \cite{b:Graphene}, a concept for scheduling DAG-based workloads in Kubernetes will be proposed in this thesis.
It will utilize well-known algorithms, such as HEFT \cite{b:HEFT} and PEFT \cite{b:PEFT}, for preplanning job execution on cluster computing resources.
To cover the required workflow management processes, such as task dependency control, a Hyperflow engine \cite{b:Hyperflow} will be used.

Both available execution strategies, with and without task clustering, are being concerned in this work.
In the process, the focus is being placed on the overall efficiency and finding the solutions most adequate to the considered situation:
% \emph{Does an additional planning phase help with workflow execution optimization?}
% \emph{If so, which scheduling algorithm proves to perform the best?}
% Moreover, both strategies, with and without task clustering enabled, are being concerned in this work.
% As ... the , challenging the 
% \emph{How does the two-step scheduling approach affect task clustering performance?}
% \emph{Compared to the optimal solution, how well does it work in terms of cluster utilization?}

% The investigation process in order to find the optimal appro revolves 
% With the 
% are various questions to consider:
% The investigation process revolves around the 

% To narrow down the scope of this thesis



\begin{itemize}
  \item Does an additional planning phase help with workflow execution optimization? If so, which scheduling algorithm proves to perform the best?
% Do the algorithms have the same relative performance in containerized computing environments?
  
  \item How does the two-step scheduling approach affect task clustering performance? Compared to the solution with optimized container CPU requests, how well does it fare?
\end{itemize}



To answer these questions, the workload executions are to be analyzed in environments with and without a workflow-aware scheduler.
The results will then be compared to determine the most adequate approach to scheduling DAG-based jobs in Kubernetes clusters. % will then be
In this work, the performance of static scheduling algorithms in containerized clusters is also to be measured and verified against the existing results from other environments.



% For a scheduling 
% The approaches will be compared with each other to determine the differences through and the best 
% In each The 
% The measured


%% Kube-sched only or with dag scheduler? How significant are differences?

%% If dag scheduler "wins" then which algorithm performs the best? Czy wyniki z paperów eksperymentalnych zostaną odzwierciedlone i uda się wskazać lepiej działający algorytm pomimo innego srodowiska wykonalnego?

%% Jak rozwiązania radzą sobie w przypadku task clusteringu? Czy uda się zmniejszyc narzut na kontenery przy użyciu dag schedulingu?

%% Jak wygląda kwestia wykorzystania zasobów? Jak wersja z dag-schedulingiem i pusta radzą sobie w porównaniu z wersją, z optymalnie ustawionymi K8s requestami zasobów?




%% -- w tej pracy we will do our best to answear all of the above questions.

%% wył0onić - emerge



% W celu odpowiedzi na postawione pytania badawcze proponujemy ->

% koncept adaptacji statycznego schedulingu DAGów do K8s

% -> inspirowany dwustopniowym podejsciem z Graphene + cytowanie ->

% -> W tym celu do planowania wykorzystamy HEFT i PEFT ->

% -> zarządanie workflowem przez Hyperflow ->


% %%% Osiągniecia pracy

% -> przeanalizowane zostaną wykonania jobów ->

% -> wyniki zostaną zestawione ze sobą -> 

% -> najlepsze rozwiązanie dla każdego z rozważanych przypadków zostanie wskazane

% -> Określony zostanie również relatywne korzyści (zyski) lepszego z rozwiązań (procentowo) ?




% To answer those questions, a concept for scheduling DAG-based workloads in Hyperflow in Kubernetes was proposed and analyzed in terms of efficiency and job completion time minimization.
% The solution was compared against the kube-scheduler to identify the strengths and weaknesses of each 

% The  The ex nalyze the differences  executions.  of scheduling thro on a Hyperflow engine, compare different approaches with each other and verify, which one tends to be most effective. 

% To answer those questions, we decided to analyze the impact of scheduling on a Hyperflow engine, compare different approaches with each other and verify, which one tends to be most effective.